\documentclass{article}
\usepackage[landscape, margin = 1cm]{geometry}
\usepackage{multicol}
\usepackage[spanish]{babel}
\usepackage[utf8]{inputenc}
\usepackage{amsfonts,amsmath,amssymb}
\usepackage{tikz}
\usetikzlibrary{decorations.pathmorphing}
\usepackage{colortbl}
\usepackage{xcolor}
\parindent0pt
\parskip2pt

%----------------------------Colores ------------------------------
\definecolor{amaranto}{rgb}{0.9, 0.17, 0.31}
\definecolor{amatista}{rgb}{0.6, 0.4, 0.8}
\definecolor{cian}{rgb}{0.13, 0.67, 0.8}
\definecolor{verdoso}{rgb}{0.0, 0.62, 0.38}
%--------------------------------------------------------------------

\pagenumbering{gobble}

%----------------------------Comandos ------------------------------
\newcommand{\arriba}[1]{\node[fancytitle] at (box.north east)
{\hspace{0.1cm} #1 \hspace{1cm}}}

\newcommand{\caja}[2]{\begin{tikzpicture}
    \node[mybox] (box){
        \begin{minipage}{0.46\textwidth}
            #2
        \end{minipage}
    };
    \arriba{#1};
    \end{tikzpicture}
    
    \vspace{0.5cm}}

\newcommand{\setcolor}[1]{\tikzstyle{mybox} = [draw=#1, fill=white, very thick, 
rectangle, rounded corners, inner sep=10pt, inner ysep=13pt, font= \sffamily]
\tikzstyle{fancytitle} =[fill=#1, text=white, left, rounded corners,
font= \sffamily \bfseries]}

\newcommand{\parte}[2]{{\huge \sffamily #1}

    \vspace{0.5cm}
    {\sffamily \textit{#2}}
    \vspace{0.2cm}}

\newcommand{\titulo}[1]{{\huge \sffamily #1}
    \vspace{0.2cm}}
%------------------------------------------------------------------------


\begin{document}

\begin{center}{\textsf{\huge{Formulario de Economía}}}
\vspace{0.5cm}

\textsf{Realizado por Daniel Jaramillo (2021)\\ En este cheat sheet recupero varias
ecuaciones de varios libros y autores que son necesarias para estudiar Economía.}

\vspace{0.5cm}
{\color{amaranto} \hrule}
\vspace{0.5cm}
\end{center}
\begin{multicols*}{2}


    \thispagestyle{empty}

\setcolor{amaranto}

\titulo{\color{amaranto}Economía Básica}

\caja{Gradiente} {
    Conjunto de derivadas parciales de primer orden de una
    función, igualadas a cero
    \begin{align}
    \nabla f (x_{1},x_{2}, \dots x_{n})= \left[ \quad \frac{\partial f(x)}{\partial x_{1}} , 
    \frac{\partial f(x)}{\partial x_{2}} ,\cdots , \frac{\partial f(x)}{\partial x_{n}} \quad \right] = 0_{n}
\end{align}}


\caja{Lagrangiano}{Es un operador/multiplicador matemático que ayuda a computar
    y resolver problemas de maximización y optimización con una función de
    restricción. Sea la función que quiero optimizar $f(x,y)$ y una restricción
    $g(x,y)$
        \begin{align}
            \mathcal{L} = f(x,y) + \lambda g(x,y)
        \end{align}
}

\caja{Gradiente del Lagrangiano}{Esta es la forma en que generalmente se
    resuelven problemas relacionados a restricciones y funciones de utilidad
\begin{align}
    \nabla \mathcal{L} = \left[ \quad \frac{\partial \mathcal{L}}{x} , 
    \frac{\partial \mathcal{L}}{y}, \frac{\partial \mathcal{L}}{\lambda} \quad
    \right] = 0_{3}
\end{align}}

\caja{Elasticidad} {Elasticidad para cualquiera dos variables w y z. Se lee como
elasticidad w de z
\begin{align}
    \varepsilon_z(w) = \frac{dz}{dw} \cdot \frac{w}{z}
\end{align}}

\caja{Utilidad de Bernoulli (Neumann y Morgenstern)} {También llamada Función de
utilidad esperada
\begin{align}
    E(x) = \sum(P_{i}x_{i}) \qquad \qquad E[U(x)] = \sum(P_{i}u(x_{i}))
\end{align}}

\caja{Coeficiente de Aversión Absoluta y relativa al riesgo}{
    Medir la concavidad de la curva de las funciones de utilidad de Bernoulli:
            \begin{align}
                C_{A}(x,u) =
                \frac{\mathrm{d}^{2}u}{\mathrm{d}x} \div
                \frac{\mathrm{d}u}{\mathrm{d}x} = \varepsilon_{u'}(x) \qquad
                \qquad C_{R}(x,u) = xC_{A}(x,u)
            \end{align}
}
    \setcolor{amatista}

\parte{\color{amatista}Teoría microeconómica}{Mas Collel hace un
análisis más profundo de la teoría económica, repasando los principios de
microeconomía desde la formalidad matemática}



\caja{Preferencias} {Sean $x_1,x_2, \dots, x_{n} \in X$, se puede establecer
    relaciones entre estos elementos, las cuales se conocen como relaciones de
    preferencia
    \begin{align}
        x \succeq x' \qquad  x\succ x' \qquad x \sim x' 
    \end{align}
    Al primero se lo lee como, $x$ \textit{se prefiere estrictamente a} $x'$, el
    segundo, \textit{se prefiere débilmente a} y el tercero, \textit{es
    indiferente a}. Estas relaciones deben cumplir los principios de continuidad
    y transitividad, los cuales se definen como
    \begin{gather}
        \forall x,y,z \in X : x \succeq y \wedge y \succeq z \therefore x \succeq z \\
        \forall x,y \in X : x \succeq y \vee y \succeq x
    \end{gather}
    siendo el principio de transitividad y de continuidad, respectivamente.
    Expresado en palabras, el primero me dice que las preferencias son
    transferibles y el segundo que un individuo siempre tiene preferencias. Esto
    es racionalidad
}

\caja{Estructuras de decisión} {Un tipo de estructura algebraica que explica
relaciones de preferencia por fuera de los principios de completitud y transitividad. 
La forma en que se define es:
    \begin{align}
        (\mathcal{B} , C(\cdot))
    \end{align}
donde $\mathcal{B}$ es un conjunto de algunas de las posibles combinaciones de
los elementos de $X$, de siempre dos o más elementos, que se llaman $B \in X$.
$C(\cdot)$ se conoce como \textbf{regla de decisión} y se define como $C: B \rightarrow x$
}

\caja{Teorema débil de la preferecia revelada}{
    Este es una conclusión que se obtiene a partir de las estructuras de decisión. 
    La expresión \textit{se revela preferible a} se expresa a través de la notación
    \begin{align}
        x \succeq^* y : x,y \in X
    \end{align}
    Y se explica a través de la regla de decisión. Sea que $\exists x,y,z \in X:
    B \in \mathcal{B} = \{(x,y), (x,z), (y,z), (x,y,z)\}$, entonces se dice que un
    consumo se revela preferible, si:
    \begin{gather}
         C(\{x,y\}) = \{x\} \therefore x \succeq^* y
    \end{gather}
}

\caja{Espacio de bienes y servicios (commodities)}{Exista un número de bienes y
    servicios disponibles en una economía, estos se representan a través de un
    espacio vectorial. Sea L el número de bienes y servicios, entonces, el
    espacio se escribe (para todo $x_{\ell} \in X$) como:
    \begin{align}
        \mathbb{R}^L = \begin{bmatrix}
            1 \\
            \vdots \\
            L 
          \end{bmatrix}
        \quad
        X = \begin{bmatrix}
        x_{1} \\
        \vdots \\
        x_{L} 
      \end{bmatrix} \in \mathbb{R}^L 
    \end{align}
}

\caja{Vector de precios}{El precio medido en términos monetarios para una 
unidad de un bien $x_{\ell} \in X$ 

    \begin{align}
        P = \begin{bmatrix}
        p_{1} \\
        \vdots \\
        p_{L} 
      \end{bmatrix} \in \mathbb{R}^L 
    \end{align}
}

\caja{Correspondencia y función de demanda Wallrasiana}{

    sean $p$ un vector de precios dados y $w$ un escalar que representa el nivel de
    riqueza de un consumidor, la función de demanda wallrasiana se define como
    $W: p,w \rightarrow \mathbb{R}^{L}_{+}$ para $X \in \mathbb{R}^{L}_{+}$ que
    es el espacio vectorial de commodities más común:
    \begin{align}
        x(p,w) = \begin{bmatrix} p_{1} \\ \vdots \\ p_{L} 
        \end{bmatrix} \cdot
        \begin{bmatrix} x_{1} \\ \vdots \\ x_{L} 
        \end{bmatrix} \leqslant w 
    \end{align}
}

\caja{Hiperplano presupuestario}{

    Sea una restricción económica asociada al problema del consumidor, se dice que
    los elementos que conforman esta restricción son:
    \begin{align}
        \{x \in \mathbb{R}^{L}_{+}: p\cdot x = w\} \quad \forall x \in B_{p,w}
    \end{align}
    Es decir, son el conjunto de cestas de consumo de un conjunto presupuestario,
    dado un vector $p$ y un escalar $w$, de manera que se cumpla la relación 
    $p\cdot x = w$. En caso de que L = 2, el hiperplano se llama recta presupuestaria
}

\caja{Ortogonalidad de $p$ y el hiperplano presupuestario}{
    
    Sea $\langle r_{p,w}\rangle = \{x \in \mathbb{R}^{L}_{+}: p\cdot x = w\}$,
    partiendo de cualquier punto $\bar{x}$ sobre $\langle r_{p,w} \rangle$ se puede trazar $\vec{p}$ de manera que:
    \begin{align}
        \vec{p}  = (\bar{x}_1 + p_1 , \bar{x}_2 + p_2 \dots, \bar{x}_L + p_L)
    \end{align}
    El producto punto se define como $|x| \cdot |p| \cdot cos(\alpha)$, donde
    $\alpha$ representa el ángulo interno entre los vectores. Si se cumple que
    $p \cdot \bar{x} = w \quad \forall \bar{x} \in \langle r_{p,w}\rangle$,
    entonces $|\cdot| = w-w=0$, por consiguiente:
    \begin{align}
        0 = |x| \cdot |p| \cdot cos(\alpha) \qquad |x|,|p| \neq 0 \therefore cos (\alpha) \overset{!}{=} 0
    \end{align}
}

\caja{Convexidad de los conjuntos presupuestarios}{
    No todos los $X$ son
    convexos, pero $\mathbb{R}^L_{+}$, uno de los subespacios presupuestarios más
    importantes, sí lo es. Un conjunto es convexo si y sólo si $\exists x'' \in X$
    tal que:
    \begin{align}
        x'' = \lambda x + (1- \lambda) x' \quad \forall x,x' \in X  \wedge \lambda \in [0,1]
    \end{align}
}

\setcolor{verdoso}


\caja{Efectos sobre la renta (wealth effects)}{
    \renewcommand*{\arraystretch}{1.8}
    Sea $x(p,w)$ una función de demanda wallrasiana, el efecto renta de la demanda pueden escribirse como:
    \begin{align}
        D_{w}x(p,w) = \frac{\partial x_{\ell}(p,w)}{\partial w} \quad \forall x_{\ell} \in X
    \end{align}
    Expresado de manera matricial, se puede decir que: 
    \begin{align}
    D_{w}x(p,w) =  \begin{bmatrix}
        \frac{\partial x_{1}(p,w)}{\partial w}  \\
        \vdots \\
        \frac{\partial x_{L}(p,w)}{\partial w}
      \end{bmatrix} \in \mathbb{R}^L 
    \end{align}
    Con esto como guía, se consiguen los conceptos de:
    \vspace{0.2cm}
    
    \textbf{Función de consumo de Engel:}
    \begin{align}
        x(\bar{p}, w) \quad E: \bar{p},w \rightarrow \mathbb{R}^L_+
    \end{align}

    \textbf{Senda de expansión:} Imagen de la función de Engels
    \begin{align}
        E_p = \{x(\bar{p},w): w>0\} \in \mathbb{R}^L_{+}
    \end{align}
    
    \textbf{Bien normal:} Si para una variación positiva de la renta, aumenta
    también la demanda de un bien $x_{\ell}$, entonces es un bien normal
    \begin{align}
        \frac{\partial x_{\ell}(p,w)}{\partial w} \geq  0
    \end{align}

    \textbf{Bien inferior:} Lo contrario a un bien normal.
    \begin{align}
        \frac{\partial x_{\ell}(p,w)}{\partial w} \leq 0
    \end{align}
}


\caja{Homogeneidad de grado cero}{Se dice
    que una función es homogenea de grado $n$ si, sea la transformación vectorial: $f: V
    \rightarrow W$ sobre el cuerpo $F$, se cumple que: 
    \begin{align}
        f(\alpha v) = \alpha^n f(v) \quad \forall \alpha \in F - \{0\} \wedge \forall v \in V
    \end{align}

    La función de demanda wallrasiana es homogénea de grado 0, es decir, que: $W: p,w \rightarrow X$:
    \begin{align}
        x(p,w) = x(\alpha p , \alpha w) \quad \forall \alpha \in \mathbb{R} - \{0\}        
    \end{align}
}

\caja{Efectos sobre el precio (precio effects)}{
    \renewcommand*{\arraystretch}{1.5}
    Sea $x(p,w)$ una función de demanda wallrasiana, el efecto precio, o efecto
    sustitución, para el precio $p_k$ y para el bien $x_\ell$ es igual a:
    \begin{align}
        D_{p}x(p,w) = \frac{\partial x_{\ell}(p,w)}{\partial p_k} \quad \forall x_\ell \in X 
    \end{align}
    Expresado de manera matricial, se puede decir que: 
    \begin{align}
    D_{p}x(p,w) =   \mathbb{J} _{x(p,w)}(p) = \begin{bmatrix}
        \nabla^T x_1\\
        \vdots \\
        \nabla^T x_2
    \end{bmatrix} = 
    \begin{bmatrix}
        \frac{\partial x_{1}}{\partial p_1} && \cdots &&  \frac{\partial x_{1}}{\partial p_L}\\
        \vdots && \ddots && \vdots \\
        \frac{\partial x_{L}}{\partial p_1} && \cdots &&  \frac{\partial x_{L}}{\partial p_L}\\
      \end{bmatrix}
    \end{align}
    Con esto como guía, se consiguen los conceptos de:
    \vspace{0.2cm}
    
    \textbf{Función de oferta:} Sea $L=2$ y se mantenga constante $p_1$ y $w$, 
     se conoce como función de oferta a la curva dada por la forma:
    \begin{align}
        x(\bar{p_1},p_2,\bar{w})
    \end{align}

    \textbf{Bien giffen:} Si la variación del precio de un bien $x_\ell$ es
    negativa $\Delta p_k <0$ y su demanda disminuye, entonces es un tipo de 
    bien giffen 
    \begin{align}
        \frac{\partial x_{\ell}(p,w)}{\partial p_\ell} > 0 
    \end{align}
    
    \textbf{Bien ordiario:} Lo contrario a los bienes giffen son los bienes ordinarios
    \begin{align}
        \frac{\partial x_{\ell}(p,w)}{\partial p_\ell} < 0 
    \end{align}

    A grandes niveles de agregación, la mayoría de bienes son ordinarios, pero a
    niveles de baja agregación, la mayoría de bienes son giffen pasado cierto umbral.
}

\caja{Ley de wallras}{Se dice
    Para cualquier nivel de precios mayor a cero y un nivel de riqueza $w$ no degenerado,
    se debe cumplir que 
    \begin{align}
        x \cdot p = w
    \end{align}
    Es decir, que el consumidor consume la totalidad de su renta. Esta ley no
tiene mucho sentido en el corto plazo ni para bienes de baja agregación, pero sí 
para consumo intertemporal y para el largo plazo}

\caja{Agregación de Cournot}{Partiendo de la ley de Wallras $p\cdot x(p,w) = w$,
    si se deriva con respecto al precio:
    \begin{align}
         \frac{\partial p_{\ell} \cdot x_{\ell}(p,w)-w}{\partial p_{k}} = p_{\ell} \cdot \frac{\partial x_\ell (p,w)}{\partial p_k} +  x_k(p,w)= 0
    \end{align}
    Si se conoce que $\varepsilon_z(w)$ es igual a la elasticidad $w$ de $z$ y
    que esta se define como $(d z / d w) \cdot (w/z)$, entonces se puede rescribir (34) como:
    \begin{align}
        \sum_{\ell=1}^{L} s_\ell \cdot \varepsilon_{\ell, k}x(p,w) = - s_k
    \end{align}
    donde $s_\ell = p_\ell \cdot (p_k/x_\ell) $ y  $s_k = p_k \cdot (x_k / x_\ell)$
    Es más sencillo de presentar la agregación de Cournot de manera vectorial:
    \begin{align}
        p \cdot D_p x(p,w) + x(p,w)^{T} = 0 
    \end{align}
}
    \thispagestyle{empty}

\setcolor{cian}

\parte{\color{cian} Economía Computacional}{Economía computacional es una rama
de la economía, la cual evalua e investiga métodos computacionales para correr e incluir
diferentes modelos económicos}

\caja{Jacobiano} {Concepto de cálculo multivariable. Matriz que utilizo para ver
linealidad local en transformaciones no lineales. Encapsula dentro de sí toda la
información de las derivadas parciales de cada función con respecto a cada
variable. Si $T: \mathbb{R}^{n} \rightarrow \mathbb{R}^{n}$ entonces el
jacobiano obligatoriamente es una matriz cuadrada. 
\begin{align}
    \mathbb{J}_{f}(x_{1},x_{2},\cdots x_{n}) = 
    \left[\begin{array}{c}
    \nabla^{T} f_{1} \\
    \vdots \\
    \nabla^{T} f_{m}
    \end{array}\right]=\left[\begin{array}{ccc}
    \dfrac{\partial f_{1}}{\partial x_{1}} & \cdots & \dfrac{\partial f_{1}}{\partial x_{n}} \\
    \vdots & \ddots & \vdots \\
    \dfrac{\partial f_{m}}{\partial x_{1}} & \cdots & \dfrac{\partial f_{m}}{\partial x_{n}}
    \end{array}\right]
\end{align}}

\caja{Forma Matricial de una ecuación lineal}{Las ecuaciones lineales se puede
escribir como el producto de una matriz de coeficientes y un vector de
incógnitas igualado a un vector conocido de constantes:
\begin{align}
    A\bar{x} = \bar{B} \qquad
    \begin{pmatrix}
    a_{11} & a_{12} \\ a_{21} & a_{22}
    \end{pmatrix}
    \cdot
    \begin{pmatrix}
    x_{1} \\ x_{2}
    \end{pmatrix}
    =
    \begin{pmatrix}
    b_{1} \\ b_{2}
    \end{pmatrix}
\end{align}}



\end{multicols*}

\end{document}
