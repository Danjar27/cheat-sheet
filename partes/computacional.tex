\thispagestyle{empty}

\setcolor{cian}

\parte{\color{cian} Economía Computacional}{Economía computacional es una rama
de la economía, la cual evalua e investiga métodos computacionales para correr e incluir
diferentes modelos económicos}

\caja{Jacobiano} {Concepto de cálculo multivariable. Matriz que utilizo para ver
linealidad local en transformaciones no lineales. Encapsula dentro de sí toda la
información de las derivadas parciales de cada función con respecto a cada
variable. Si $T: \mathbb{R}^{n} \rightarrow \mathbb{R}^{n}$ entonces el
jacobiano obligatoriamente es una matriz cuadrada. 
\begin{align}
    \mathbb{J}_{f}(x_{1},x_{2},\cdots x_{n}) = 
    \left[\begin{array}{c}
    \nabla^{T} f_{1} \\
    \vdots \\
    \nabla^{T} f_{m}
    \end{array}\right]=\left[\begin{array}{ccc}
    \dfrac{\partial f_{1}}{\partial x_{1}} & \cdots & \dfrac{\partial f_{1}}{\partial x_{n}} \\
    \vdots & \ddots & \vdots \\
    \dfrac{\partial f_{m}}{\partial x_{1}} & \cdots & \dfrac{\partial f_{m}}{\partial x_{n}}
    \end{array}\right]
\end{align}}

\caja{Forma Matricial de una ecuación lineal}{Las ecuaciones lineales se puede
escribir como el producto de una matriz de coeficientes y un vector de
incógnitas igualado a un vector conocido de constantes:
\begin{align}
    A\bar{x} = \bar{B} \qquad
    \begin{pmatrix}
    a_{11} & a_{12} \\ a_{21} & a_{22}
    \end{pmatrix}
    \cdot
    \begin{pmatrix}
    x_{1} \\ x_{2}
    \end{pmatrix}
    =
    \begin{pmatrix}
    b_{1} \\ b_{2}
    \end{pmatrix}
\end{align}}